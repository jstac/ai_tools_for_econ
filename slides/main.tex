\documentclass[
    xcolor={svgnames,dvipsnames},
    hyperref={colorlinks, citecolor=DeepPink4, linkcolor=DarkRed, urlcolor=DarkBlue}
    ]{beamer}  % for hardcopy add 'trans'

\mode<presentation>
{
  \usetheme{Singapore}
  % or ...
  \setbeamercovered{transparent}
  % or whatever (possibly just delete it)
}

%\usefonttheme{professionalfonts}
%\usepackage[english]{babel}
% or whatever
%\usepackage[latin1]{inputenc}
% or whatever
%\usepackage{times}
%\usepackage[T1]{fontenc}
% Or whatever. Note that the encoding and the font should match. If T1
% does not look nice, try deleting the line with the fontenc.

%\usepackage{fontspec}
%\setmonofont{CMU Typewriter Text}
%\setmonofont{Consolas}

\usepackage{fontspec} 
%\usepackage[xcharter]{newtxmath}
%\setmainfont{XCharter}
\usepackage{unicode-math}
%\setmathfont{XCharter-Math.otf}
\setmonofont{DejaVu Sans Mono}[Scale=MatchLowercase] % provides unicode characters 



%%%%%%%%%%%%%%%%%%%%%% start my preamble %%%%%%%%%%%%%%%%%%%%%%

\addtobeamertemplate{navigation symbols}{}{%
    \usebeamerfont{footline}%
    \usebeamercolor[fg]{footline}%
    \hspace{1em}%
    \insertframenumber/\inserttotalframenumber
}


\usepackage{graphicx}
\usepackage{amsmath, amssymb, amsthm}
\usepackage{bbm}
\usepackage{mathrsfs}
\usepackage{xcolor}
\usepackage{fancyvrb}


% Quotes at start of chapters / sections
\usepackage{epigraph}  
%\renewcommand{\epigraphflush}{flushleft}
%\renewcommand{\sourceflush}{flushleft}
\renewcommand{\epigraphwidth}{6in}

%% Fonts

%\usepackage[T1]{fontenc}
\usepackage{mathpazo}
%\usepackage{fontspec}
%\defaultfontfeatures{Ligatures=TeX}
%\setsansfont[Scale=MatchLowercase]{DejaVu Sans}
%\setmonofont[Scale=MatchLowercase]{DejaVu Sans Mono}
%\setmathfont{Asana Math}
%\setmainfont{Optima}
%\setmathrm{Optima}
%\setboldmathrm[BoldFont={Optima ExtraBlack}]{Optima Bold}

% Some colors

\definecolor{aquamarine}{RGB}{69,139,116}
\definecolor{midnightblue}{RGB}{25,25,112}
\definecolor{darkslategrey}{RGB}{47,79,79}
\definecolor{darkorange4}{RGB}{139,90,0}
\definecolor{dogerblue}{RGB}{24,116,205}
\definecolor{blue2}{RGB}{0,0,238}
\definecolor{bg}{rgb}{0.95,0.95,0.95}
\definecolor{DarkOrange1}{RGB}{255,127,0}
\definecolor{ForestGreen}{RGB}{34,139,34}
\definecolor{DarkRed}{RGB}{139, 0, 0}
\definecolor{DarkBlue}{RGB}{0, 0, 139}
\definecolor{Blue}{RGB}{0, 0, 255}
\definecolor{Brown}{RGB}{165,42,42}


\setlength{\parskip}{1.5ex plus0.5ex minus0.5ex}

%\renewcommand{\baselinestretch}{1.05}
%\setlength{\parskip}{1.5ex plus0.5ex minus0.5ex}
%\setlength{\parindent}{0pt}

% Typesetting code
\definecolor{bg}{rgb}{0.95,0.95,0.95}
\usepackage{minted}
\setminted{mathescape, frame=lines, framesep=3mm}
\usemintedstyle{friendly}
%\newminted{python}{}
%\newminted{c}{mathescape,frame=lines,framesep=4mm,bgcolor=bg}
%\newminted{java}{mathescape,frame=lines,framesep=4mm,bgcolor=bg}
%\newminted{julia}{mathescape,frame=lines,framesep=4mm,bgcolor=bg}
%\newminted{ipython}{mathescape,frame=lines,framesep=4mm,bgcolor=bg}


\newcommand{\Fact}{\textcolor{Brown}{\bf Fact. }}
\newcommand{\Facts}{\textcolor{Brown}{\bf Facts }}
\newcommand{\keya}{\textcolor{turquois4}{\bf Key Idea. }}
\newcommand{\Factnodot}{\textcolor{Brown}{\bf Fact }}
\newcommand{\Eg}{\textcolor{ForestGreen}{Example. }}
\newcommand{\Egs}{\textcolor{ForestGreen}{Examples. }}
\newcommand{\Ex}{{\bf Ex. }}



\renewcommand{\theFancyVerbLine}{\sffamily
    \textcolor[rgb]{0.5,0.5,1.0}{\scriptsize {\arabic{FancyVerbLine}}}}

\newcommand{\navy}[1]{\textcolor{DarkBlue}{\bf #1}}
\newcommand{\brown}[1]{\textcolor{Brown}{\sf #1}}
\newcommand{\green}[1]{\textcolor{ForestGreen}{\sf #1}}
\newcommand{\blue}[1]{\textcolor{Blue}{\sf #1}}
\newcommand{\emp}[1]{\textcolor{DarkOrange1}{\bf #1}}
\newcommand{\red}[1]{\textcolor{Red}{\bf #1}}

% Symbols, redefines, etc.

\newcommand{\code}[1]{\texttt{#1}}

\newcommand{\argmax}{\operatornamewithlimits{argmax}}
\newcommand{\argmin}{\operatornamewithlimits{argmin}}

\DeclareMathOperator{\cl}{cl}
\DeclareMathOperator{\interior}{int}
\DeclareMathOperator{\Prob}{Prob}
\DeclareMathOperator{\determinant}{det}
\DeclareMathOperator{\trace}{trace}
\DeclareMathOperator{\Span}{span}
\DeclareMathOperator{\rank}{rank}
\DeclareMathOperator{\cov}{cov}
\DeclareMathOperator{\corr}{corr}
\DeclareMathOperator{\var}{var}
\DeclareMathOperator{\mse}{mse}
\DeclareMathOperator{\se}{se}
\DeclareMathOperator{\row}{row}
\DeclareMathOperator{\col}{col}
\DeclareMathOperator{\range}{rng}
\DeclareMathOperator{\dimension}{dim}
\DeclareMathOperator{\bias}{bias}


% mics short cuts and symbols
\newcommand{\st}{\ensuremath{\ \mathrm{s.t.}\ }}
\newcommand{\setntn}[2]{ \{ #1 : #2 \} }
\newcommand{\cf}[1]{ \lstinline|#1| }
\newcommand{\fore}{\therefore \quad}
\newcommand{\tod}{\stackrel { d } {\to} }
\newcommand{\toprob}{\stackrel { p } {\to} }
\newcommand{\toms}{\stackrel { ms } {\to} }
\newcommand{\eqdist}{\stackrel {\textrm{ \scriptsize{d} }} {=} }
\newcommand{\iidsim}{\stackrel {\textrm{ {\sc iid }}} {\sim} }
\newcommand{\1}{\mathbbm 1}
\newcommand{\dee}{\,{\rm d}}
\newcommand{\given}{\, | \,}
\newcommand{\la}{\langle}
\newcommand{\ra}{\rangle}

\newcommand{\boldA}{\mathbf A}
\newcommand{\boldB}{\mathbf B}
\newcommand{\boldC}{\mathbf C}
\newcommand{\boldD}{\mathbf D}
\newcommand{\boldM}{\mathbf M}
\newcommand{\boldP}{\mathbf P}
\newcommand{\boldQ}{\mathbf Q}
\newcommand{\boldI}{\mathbf I}
\newcommand{\boldX}{\mathbf X}
\newcommand{\boldY}{\mathbf Y}
\newcommand{\boldZ}{\mathbf Z}

\newcommand{\bSigmaX}{ {\boldsymbol \Sigma_{\hboldbeta}} }
\newcommand{\hbSigmaX}{ \mathbf{\hat \Sigma_{\hboldbeta}} }

\newcommand{\RR}{\mathbbm R}
\newcommand{\NN}{\mathbbm N}
\newcommand{\PP}{\mathbbm P}
\newcommand{\EE}{\mathbbm E \,}
\newcommand{\XX}{\mathbbm X}
\newcommand{\ZZ}{\mathbbm Z}
\newcommand{\QQ}{\mathbbm Q}

\newcommand{\fF}{\mathcal F}
\newcommand{\dD}{\mathcal D}
\newcommand{\lL}{\mathcal L}
\newcommand{\gG}{\mathcal G}
\newcommand{\hH}{\mathcal H}
\newcommand{\nN}{\mathcal N}
\newcommand{\pP}{\mathcal P}




\title{An AI-Driven Revolution in Scientific Computing}



\author{John Stachurski}


\date{RBA Nov 2024}


\begin{document}

\begin{frame}
  \titlepage
\end{frame}



\begin{frame}
    \frametitle{Topics}

    \brown{Part 1}: Slides

    \begin{itemize}
        \item AI-driven scientific computing
        \vspace{0.5em}
        \item Applications  
        \vspace{0.5em}
    \end{itemize}

    \vspace{0.5em}
    \vspace{0.5em}
    \brown{Part 2}: Hands on coding


    \vspace{0.5em}

    \begin{center}
        \url{https://github.com/QuantEcon/rba_workshop_2024}
    \end{center}


\end{frame}



\begin{frame}
    \frametitle{AI-driven scientific computing}

    AI is changing the world

    \begin{itemize}
        \item image processing / computer vision
        \vspace{0.5em}
        \item speech recognition, translation
        \vspace{0.5em}
        \item scientific knowledge discovery
        \vspace{0.5em}
        \item forecasting and prediction 
        \vspace{0.5em}
        \item generative AI
    \end{itemize}

    \pause

        \vspace{0.5em}
        \vspace{0.5em}
        \vspace{0.5em}
    Plus killer drones, skynet, etc.\ldots

    
\end{frame}

\begin{frame}
    
    Projected spending on AI in 2024:

    \begin{itemize}
        \item Google: \$50 billion
        \vspace{0.5em}
        \item Microsoft: \$60 billion
        \vspace{0.5em}
        \item Meta: \$40 billion
        \vspace{0.5em}
        \item etc.
    \end{itemize}

    Investments in software and hardware

        \vspace{0.5em}
        \vspace{0.5em}
    Software component is mainly open source
        \vspace{0.5em}
        \vspace{0.5em}

    What does this software do?

\end{frame}





\begin{frame}
    \frametitle{Deep learning in two slides}
    
    Aim: approximate an unknown functional relationship
    %
    \begin{equation*}
        y = f(x)
        \qquad (x \in \RR^k, \; y \in \RR)
    \end{equation*}

    \Egs
    %
    \begin{itemize}
        \item $x = $ cross section of returns, $y = $ return on oil futures tomorrow
        \vspace{0.5em}
        \item $x = $ weather sensor data, $y = $ max temp tomorrow
    \end{itemize}
        \vspace{0.5em}
        \vspace{0.5em}

    Problem:

    \begin{itemize}
        \item observe $(x_i, y_i)_{i=1}^n$ and seek $f$ such that $y_{n+1}
            \approx f(x_{n+1})$
    \end{itemize}


\end{frame}


\begin{frame}

    Nonlinear regression: choose model $\{f_\theta\}_{\theta \in \Theta}$ and minimize the empirical loss
    %
    \begin{equation*}
        \ell(\theta) := \sum_{i=1}^n (y_i - f_\theta(x_i))^2
        \quad \st \quad \theta \in \Theta
    \end{equation*}


    \pause
    \vspace{0.5em}
    In the case of ANNs, we consider all $f_\theta$ having the form
    %
    \begin{equation*}
        f_\theta
        = \sigma \circ A_{m} 
            \circ \cdots \circ \sigma \circ A_{2}  \circ \sigma \circ A_{1}
    \end{equation*}
    %
    where
    %
    \begin{itemize}
        \item $A_{j} x = W_j x + b_j $ is an affine map 
        \vspace{0.5em}
        \begin{itemize}
            \item \texttt{output = dot(kernel, input) + bias}
        \end{itemize}
        \vspace{0.5em}
        \item $\sigma$ is a nonlinear ``activation'' function
    \end{itemize}

\end{frame}


\begin{frame}
    

    Minimizing a smooth loss functions  -- what algorithm?
    
    \begin{figure}
       \begin{center}
        \scalebox{0.15}{\includegraphics[trim={0cm 0cm 0cm 0cm},clip]{gdi.png}}
       \end{center}
    \end{figure}

    Source: \url{https://danielkhv.com/}

\end{frame}


\begin{frame}

    Deep learning: $\theta \in \RR^d$ where $d = ?$
    
    \begin{figure}
       \begin{center}
        \scalebox{0.14}{\includegraphics[trim={0cm 0cm 0cm 0cm},clip]{loss2.jpg}}
       \end{center}
    \end{figure}

    Source: \url{https://losslandscape.com/gallery/}

\end{frame}




\begin{frame}
    \frametitle{How does it work?}
    
    Why is it possible to minimize over $\theta \in \RR^d$ when $d=10^{12}$ ?!?

        \vspace{0.5em}
        \vspace{0.5em}
        \vspace{0.5em}
        \vspace{0.5em}
        \pause

    Core elements
    %
    \begin{itemize}
        \item automatic differentiation (for \underline{gradient} descent)
        \vspace{0.5em}
        \item parallelization (GPUs or TPUs)
        \vspace{0.5em}
        \item Compilers / JIT-compilers
    \end{itemize}

\end{frame}

\begin{frame}[fragile]
    \frametitle{Automatic differentiation}

    \vspace{0.5em}
    ``Exact numerical'' differentiation
    
    \begin{minted}{python}

def loss(θ, x, y):
  return jnp.sum((y - f(θ, x))**2)

loss_gradient = grad(loss)

    \end{minted}

    \vspace{0.5em}
    \vspace{0.5em}
    \vspace{0.5em}
    Now use gradient descent\ldots


\end{frame}


\begin{frame}[fragile]
    \frametitle{Parallelization}

    
    \begin{minted}{python}
outputs = pmap(f, data)  
    \end{minted}

    \vspace{0.5em}
    \vspace{0.5em}
    \begin{itemize}
        \item multithreading over GPU cores (how many?)
        \vspace{0.5em}
        \item multiprocessing over accelerators in a GPU farm / supercomputing
            cluster (how many?)
    \end{itemize}

\end{frame}


\begin{frame}
    
    \begin{figure}
       \begin{center}
        \scalebox{0.18}{\includegraphics[trim={0cm 0cm 0cm 0cm},clip]{dgx.png}}
       \end{center}
    \end{figure}


\end{frame}


\begin{frame}[fragile]
    \frametitle{Just-in-time compilers}

    \vspace{0.5em}
    
    \begin{minted}{python}
@jit
def f(x):
    return jnp.sin(x) - jnp.cos(x**2)
    \end{minted}

    \vspace{0.5em}
    Advantages over AOT compilers:

    \begin{itemize}
        \item cleaner code
    \vspace{0.5em}
        \item more portable
    \vspace{0.5em}
        \item automatic parallelization (same code for CPUs / GPUs)
    \end{itemize}

\end{frame}


\begin{frame}
    
    Advantages over NumPy / MATLAB

    \vspace{0.5em}
    \begin{itemize}
        \item can specialize machine code based on parameter types / shapes
        \vspace{0.5em}
        \item automatically matches tasks with accelerators (GPU / TPU)
        \vspace{0.5em}
        \item fuses array operations for speed and memory efficiency
    \end{itemize}

\end{frame}


\begin{frame}
    \frametitle{Platforms}
    
    Platforms that support AI / deep learning:

    \vspace{0.5em}
    \begin{itemize}
        \item Tensorflow
        \vspace{0.5em}
        \item \brown{PyTorch} (Llama, ChatGPT)
        \vspace{0.5em}
        \item \brown{Google JAX} (Gemini, DeepMind)
        \vspace{0.5em}
        \item Keras (backends $=$ JAX, PyTorch)
        \vspace{0.5em}
        \item Mojo? (Modular (Python))
        \vspace{0.5em}
        \item MATLAB? 
    \end{itemize}

\end{frame}




\begin{frame}
    
    Popularity

    
    \begin{figure}
       \begin{center}
        \scalebox{0.32}{\includegraphics[trim={0cm 0cm 0cm 0cm},clip]{c.png}}
       \end{center}
    \end{figure}


\end{frame}


\begin{frame}
    \frametitle{Focus on JAX}

    \url{https://jax.readthedocs.io/en/latest/}
    
            \vspace{0.5em}
    \begin{itemize}
        \item \brown{J}ust-in-time compilation
            \vspace{0.5em}
        \item \brown{A}utomatic differentiation
            \vspace{0.5em}
        \item \brown{X}ccelerated linear algebra
    \end{itemize}

            \vspace{0.5em}
            \vspace{0.5em}

\end{frame}


\begin{frame}[fragile]
    
    \vspace{-1em}
    \begin{minted}{python}
import jax.numpy as jnp
from jax import grad, jit

def f(θ, x):
  for W, b in θ:
    w = W @ x + b
    x = jnp.tanh(w)  
  return x

def loss(θ, x, y):
  return jnp.sum((y - f(θ, x))**2)

grad_loss = jit(grad(loss))  # Now use gradient descent 
    \end{minted}

\end{frame}


\begin{frame}
    
    \Eg AlphaFold3 (Google JAX)

        \vspace{0.5em}
    \textbf{Highly accurate protein structure prediction with AlphaFold}

        \vspace{0.5em}
    John Jumper, Richard Evans, Alexander Pritzel, Tim Green, Michael Figurnov,
    Olaf Ronneberger, Kathryn Tunyasuvunakool,\ldots 

        \vspace{0.5em}
    \underline{Nature} Vol.\ 596 (2021)

    \vspace{0.5em}
    \vspace{0.5em}
    \vspace{0.5em}
    \vspace{0.5em}
    \begin{itemize}
        \item Citation count $= ~30$K
        \vspace{0.5em}
        \item Nobel Prize in Chemistry 2024
    \end{itemize}

\end{frame}


\begin{frame}
    \frametitle{AI tools for economic modeling}


    Let's say that you want to do computational macro rather than deep learning

    \vspace{0.5em}
    Can these new AI tools be applied?

    \pause

    \vspace{0.5em}
    \vspace{0.5em}
    Yes!

    \begin{itemize}
        \item fast matrix algebra
        \vspace{0.5em}
        \item fast solutions to linear systems
        \vspace{0.5em}
        \item fast nonlinear system solvers
        \vspace{0.5em}
        \item fast optimization, etc.
    \end{itemize}


\end{frame}


\begin{frame}

    Advantages of JAX (vs PyTorch / Numba / etc.) for economists:
    %
    \begin{itemize}
        \item exposes low level functions
            \vspace{0.5em}
        \item elegant functional programming style -- close to maths
            \vspace{0.5em}
        \item elegant autodiff tools
            \vspace{0.5em}
        \item array operations follow standard NumPy API
            \vspace{0.5em}
        \item automatic parallelization
            \vspace{0.5em}
        \item same code, multiple backends (CPUs, GPUs, TPUs)
    \end{itemize}

\end{frame}


\begin{frame}
    \frametitle{Case Study}

    The CBC uses the ``overborrowing'' model of Bianchi (2011)

    \begin{itemize}
        \item credit constraint loosens during booms
        \item bad shocks $\to$ sudden stops
    \end{itemize}

    \vspace{0.5em}
    CBC implementation in MATLAB 

    \begin{itemize}
        \item runs on \$10,000 mainframe with 356 CPUs and 1TB RAM
        \item runtime $=$ 12 hours
    \end{itemize}

    \pause
    \vspace{0.5em}
    Rewrite in Python + Google JAX

    \begin{itemize}
        \item runs on \$400 gaming GPU with 10GB RAM
        \item runtime $=$ 4.17 seconds
    \end{itemize}

\end{frame}


\begin{frame}
    \frametitle{Live coding}

    See notebooks in
    
    \begin{center}
        \url{https://github.com/QuantEcon/rba_workshop_2024}
    \end{center}

    \vspace{0.5em}
    Steps
    %
    \begin{enumerate}
    \item Go to Google Colab (\url{https://colab.google/})
    \vspace{0.5em}
    \item Open notebook $\to$ GitHub $\to$ quantecon $\to$ \texttt{rba\_workshop\_2024}
        $\to$ select notebook
    \vspace{0.5em}
    \item Edit $\to$ Notebook settings $\to$ select a GPU
    \vspace{0.5em}
    \item Shift-enter to run each cell
    \end{enumerate}

\end{frame}


\begin{frame}
    \frametitle{AI for Coding}

    Intro 

\end{frame}


\begin{frame}

    Claude:
    
    ``I'm definitely stronger with Python than MATLAB.

    While I'm knowledgeable about both languages, my capabilities with Python
    are more comprehensive. I have deeper familiarity with Python's extensive
    ecosystem of libraries, frameworks, and modern development practices. I can
    more confidently help with advanced Python topics, debugging complex Python
    code, and implementing Python best practices.''

\end{frame}


\begin{frame}
    
    Claude:

    ``I'm definitely stronger with Python than Julia.

    Python is one of my most proficient languages - I have deep familiarity with
    its syntax, libraries, frameworks, and best practices across many domains
    including data science, web development, machine learning, and
    general-purpose programming.

    While I understand Julia's syntax and core concepts, my expertise with it
    isn't as comprehensive as with Python.''

\end{frame}

\end{document}


